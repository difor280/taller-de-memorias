\documentclass{article}
\usepackage[utf8]{inputenc}
\usepackage[spanish]{babel}
\usepackage{listings}
\usepackage{graphicx}
\graphicspath{ {images/} }
\usepackage{cite}

\begin{document}

\begin{titlepage}
    \begin{center}
        \vspace*{1cm}
            
        \Huge
        \textbf{Informática II}
            
        \vspace{0.5cm}
        \LARGE
        Taller - Nociones de la memoria del computador
            
        \vspace{1.5cm}
            
        \textbf{Diego Fernando Urbano Palma}
            
        \vfill
            
        \vspace{0.8cm}
            
        \Large
        Despartamento de Ingeniería Electrónica y Telecomunicaciones\\
        Universidad de Antioquia\\
        Medellín\\
        Marzo de 2021
            
    \end{center}
\end{titlepage}

\tableofcontents
\newpage
\section{ Defina que es la memoria del computador.}\label{intro}

la memoria del computador es donde se almacena temporalmente la
información que se encuentra en procesamiento, la analogia que pone el documento es que la memori del computador es como un escritorio donde se dejan los archivos que se sacan del cajon (disco duro), y donde el empleado (microporcesador) analiza documentos uno por uno hasta terniar con ellos repitiendolo una y otra vez. 
Se le llama memorio a todo dispositivo de almacenamiento electronico pero usualmente es un termino para los dispositivos de almacenamiento temporal y rapido 

\section{Mencione los tipos de memoria que conoce y haga una pequeña descripción de cada tipo.} \label{contenido}

\textbf{cache L1,L2,L3: }estas memorias estan unicadas dentro del procesador y son las memorias mas rapidas de todas pero eso tambien son las mas costos, su nivel de velocidad disminuye conforme su numeracion aumenta pero cada aunmento agrega mas almacenamiento por ende entre mas numeracion tiene mas capacidad tren, solo que su capacidad no es tan grande como para guardar programas.

\textbf{RAM:} es la memoria que se encarga de guardar lo que hace en el procesador y los programas que necesite de momento, esta memoria es de corto-mediano plazo, ya que al apagar el computador todo lo que  esta en la memoria se borra la ventaja que tiene con respecto al disco duro es que es mucho mas rapida, por eso es mas eficiente para trabajar con el procesador     

\textbf{MEMORIA VIRTUAL: } es un espacio que tiene el disco duro para guardar la informacion menos necesaria pero que si la necesitan, estan listas para salir solo que usa tan poco que se guarda en esta porcion del disco duro 

\textbf{DISCO DURO: } es la memoria que se encarga de guardar las ordenes de al ram, esta es a largo plazo ya que no se borra cuando el equipo se apaga, aca  es donde se instalas todo lo  del computador,  programas, en conclusion todo el software, esta memoria es muy lenta como para trabajar con el procesador. 

\textbf{MEMORIA USB: } esta es una memoria externa del computador, la cual se puede transportar, guarda los datos a largo plano, osea, no borra su contenido cuando se desconecta del equipo, las desventaja con el disco duro, es su bajo capacidad, procesamiento de datos o velocidad de interaccion  

\section{Describa la manera como se gestiona la memoria en un computador.}

la memoria es un lugar donde se guarda termoralmente los programas y modificaciones ya que es mas rapida que el disco duro pero guarda los archivos de forma termporal, lo que hace que el procesador este en continuia comunicacion con ella para guardar los procesos que el va ejerciendo, tambien mantiene en contanto con el disco duro para mover informacion del disco duro a la memoria y asi progresivamente asi que la memoria guarda ordenes hasta que se cumple, y programas hasta todo otra vez este guardado-clonado en el disco duro, ya cuando todo esta guardado, se borra el pograma y lista para resivir mas ordenes 

\section{¿Qué hace que una memoria sea más rápida que otra? ¿Por qué esto es importante?}

la candidad de transistores, uanque no estoy seguro y la contuctividad de los materiasles. Esto es importante porque entre mas rapida sea una memoria mas fluido sera la interaccion con el usuario y mas rapido sera el intercambio de informacion 


\end{document}
