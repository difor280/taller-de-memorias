\documentclass{article}
\usepackage[utf8]{inputenc}
\usepackage[spanish]{babel}
\usepackage{listings}
\usepackage{graphicx}
\graphicspath{ {images/} }
\usepackage{cite}

\begin{document}

\begin{titlepage}
    \begin{center}
        \vspace*{1cm}
            
        \Huge
        \textbf{Informática II}
            
        \vspace{0.5cm}
        \LARGE
        Taller - Nociones de la memoria del computador
            
        \vspace{1.5cm}
            
        \textbf{Diego Fernando Urbano Palma}
            
        \vfill
            
        \vspace{0.8cm}
            
        \Large
        Despartamento de Ingeniería Electrónica y Telecomunicaciones\\
        Universidad de Antioquia\\
        Medellín\\
        Marzo de 2021
            
    \end{center}
\end{titlepage}

\tableofcontents
\newpage
\section{ Defina que es la memoria del computador.}\label{intro}

la memoria del computador es un conjuto de hardware encargado que almacenar datos a corto y largo plazo

\section{Mencione los tipos de memoria que conoce y haga una pequeña descripción de cada tipo.} \label{contenido}

\textbf{RAM:} es la memoria que se encarga de guardar lo que hace en el procesador, esta memoria es de corto-mediano plazo, ya que al apagar el computador todo lo que  esta en la memoria se borra    

\textbf{DISCO DURO: } es la memoria que se encarga de guardar las ordenes de al ram, esta es a largo plazo ya que no se borra cuando el equipo se apaga, aca  es donde se instalan las cosas importantes del computador, como son fiches, programas, en conclusion todo el software. 

\textbf{MEMORIA USB: } esta es una memoria externa del computador, la cual se puede transportar, guarda los datos a largo plano, osea, no borra su contenido cuando se desconecta del equipo, la unica desventaja con el disco duro, es su bajo capacidad y en ocasiones su procesamiento de datos 

\section{Describa la manera como se gestiona la memoria en un computador.}

la \textbf{RAM} es la que guarda toda la informacion del procesador, programas que se esten corriendo en ese momento,etc, de allí pasamos a al \textbf{disco duro}, donde se almacena todo lo que tiene que ver con programas y se saca informacion antigua que el procesador requiera 

\section{¿Qué hace que una memoria sea más rápida que otra? ¿Por qué esto es importante?}

la candidad de transistores, uanque no estoy seguro y la contuctividad de los materiasles. Esto es importante porque entre mas rapida sea una memoria mas fluido sera la interaccion con el usuario y mas rapido sera el intercambio de informacion 


\end{document}
